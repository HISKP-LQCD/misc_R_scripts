\documentclass[10pt,a4paper]{article}
\usepackage[utf8x]{inputenc}
\usepackage{ucs}
\usepackage{amsmath}
\usepackage{amsfonts}
\usepackage{amssymb}

\author{Bartosz Kostrzewa}
\title{Documentation: meson\_2pt analysis tools}
\begin{document}
This document describes the design of the library of R scripts for the analysis of meson 2 point functions.
The software is structured as a set of modules which are to be called from within a user-defined driver function for which an example is provided with the tool.
This driver function loads .Rdata files produced by the {\ttfamily analysis\_conn\_meson\_2pt} script and organizes the analysis results into a list of lists of the data type {\ttfamily meson\_obs}.

\begin{section}{{\ttfamily meson\_obs} data type}
This is a list of lists, the outermost index of which is numbered.
Each list element contains the elements: {\tt name, texlabel, m.val, m.sea, mean, err, boot}

\begin{table}[h]
\begin{tabular}{|p{0.25\linewidth}|p{0.65\linewidth}|}
\hline
\multicolumn{2}{|c|}{ {\tt meson\_obs} data type } \\
\hline \hline \textbf{member name} & \textbf{description} \\ \hline
{\tt name} & string, name of the quantity being stored \\ 
{\tt texlabel}  & string, appropriately escaped (La)TeX code for labelling this quantity  \\ 
{\tt m.val} & numeric vector of valence quark masses that this quantity depends on \\
{\tt m.sea} & numeric vector of sea quark masses that this quantity depends on \\
{\tt mean} & single number, mean of the quantity \\
{\tt err} & single number, error on the quantity \\
{\tt boot} & numeric vector of bootstrap samples of this quantity \\
\hline 
\end{tabular}
\end{table}

\begin{subsection}{{\tt select.meson\_obs} function}
The flexibility of the list data type has the drawback that R's nice indexing methods fail to work here.
This function is used to extract a subset from the complete list.

\begin{table}[h]
\begin{tabular}{|p{0.25\linewidth}|p{0.65\linewidth}|}
\hline
\multicolumn{2}{|c|}{ {\tt select.meson\_obs} } \\
\hline \hline \textbf{argument name} & \textbf{description} \\
\hline {\tt meson\_obs} & object of the {\tt meson\_obs type} \\ 
{\tt by} & string, name of the field that the selection should be based on \\ 
{\tt filter} & depends on data type of the field specified by {\tt by}, the value that should be matched \\ 
\hline \hline
\textbf{return value} & if there are no matches then this returns an empty list, otherwise an object of class {\tt meson\_obs} \\
\hline
\end{tabular}
\end{table}

\end{subsection}

\begin{subsection}{ {\tt as.data.frame.meson\_obs} function }
{\tt select.meson\_obs} can be used to extract a consistent set of data; i.e., one where each list element has named elements of the same length, say e.g. the same number of valence masses).
This function can be used to transform the list into a data frame which can then be passed on to actual analysis routines.

\begin{table}[h]
\begin{tabular}{|p{0.25\linewidth}|p{0.65\linewidth}|}
\hline
\multicolumn{2}{|c|}{ {\tt as.data.frame.meson\_obs} } \\
\hline \hline \textbf{argument name} & \textbf{description} \\ \hline
\hline 
\textbf{return value} & xxx \\
\hline
\end{tabular}
\end{table}

\end{subsection}

\begin{section}{The F.E.S. utility functions}
\begin{subsection}{ {\tt fes\_fit\_linear} }
Routine which carries out a linear fit in an arbitrary number of variables of the form:

\begin{equation}
y=a_1 x_1 + a_2 x_2 + \ldots + a_n x_n + c.
\end{equation}

The fit function is constructed automatically based on the number of columns in the {\tt dat} argument.

\begin{table}[h]
\begin{tabular}{|p{0.25\linewidth}|p{0.65\linewidth}|}
\hline
\multicolumn{2}{|c|}{ {\tt as.data.frame.meson\_obs} } \\
\hline \hline \textbf{argument name} & \textbf{description} \\ \hline
{\tt dat} & List of data frames, the outermost index of which enumerates the bootstrap samples.
The data frames must have column names: {\tt y, x1, x2, \ldots, xN, weight}, in this order. \\
\hline 
\textbf{return value} & Object of the classes {\tt fesfit} and {\tt fesfit\_linear}. \\
\hline
\end{tabular}
\end{table}

\end{subsection}

\begin{subsection}{ {\tt fesfit\_linear} data type }
Return value data type of the {\tt fes\_fit\_linear} function consisting of the elements: {\tt fit}, {\tt n} and {\tt model}.

\begin{table}[h]
\begin{tabular}{|p{0.25\linewidth}|p{0.65\linewidth}|}
\hline
\multicolumn{2}{|c|}{ {\tt fesfit\_linear} data type } \\
\hline \hline \textbf{member name} & \textbf{description} \\ \hline
{\tt fit} & numbered list of {\tt nls} objects \\ 
{\tt n}  & number of elements in {\tt fit}, specifies for how many bootstrap samples the given fit worked \\ 
{\tt model} & string of the form {\tt y $\sim$ a1*x2 + a2*x2 + \ldots + c} which can be passed to {\tt as.formula} \\
\hline 
\end{tabular}
\end{table}


\end{subsection}

\begin{subsection}{ {\tt fes\_extrapolate} }
\end{subsection}

\begin{subsection}{ {\tt fes\_solve} }
\end{subsection}

\end{section} 


\end{section}

\end{document}